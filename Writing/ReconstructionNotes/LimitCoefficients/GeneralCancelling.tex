\documentclass[10pt]{article}

  \usepackage{pgfplots}
\pgfplotsset{compat=newest}
%% the following commands are needed for some matlab2tikz features
\usetikzlibrary{plotmarks}
\usetikzlibrary{arrows.meta}
\usepgfplotslibrary{patchplots}
\usepackage{grffile}
\usepackage{amsmath}
\usepackage{lineno}


%\usepackage{fullpage}
\usepackage[top=1in, bottom=1in, left=0.8in, right=1in]{geometry}
\usepackage{multicol}
\usepackage{caption}
\usepackage{subcaption}
\usepackage{hyperref}
\usepackage{xcolor}
\usepackage{graphicx,psfrag}
\usepackage[pdf]{pstricks}

\definecolor{lightblue}{rgb}{.80,.9,1}
\newcommand{\hl}[1]
    {\par\colorbox{lightblue}{\parbox{\linewidth}{#1}}}

\newcommand{\defn}{\stackrel{\textrm{\scriptsize def}}{=}}

\setlength{\columnsep}{0.1pc}

\title{Arbitrarily High Order TVD Reconstruction Polynomials}
\author{Jordan Pitt -- \texttt{jordan.pitt@anu.edu.au}}

% TIME ON EVERY PAGE AS WELL AS THE FILE NAME
\usepackage{fancyhdr}
\usepackage{currfile}
\usepackage[us,12hr]{datetime} % `us' makes \today behave as usual in TeX/LaTeX
\fancypagestyle{plain}{
\fancyhf{}
\rfoot{\emph{\footnotesize \textcopyright  Serre Notes by  J. Pitt, C. Zoppou and S. Roberts.}
 \\ File Name: {\currfilename} \\ Date: {\ddmmyyyydate\today} at \currenttime}
\lfoot{Page \thepage}
\renewcommand{\headrulewidth}{0pt}}
\pagestyle{plain}

\definecolor{mycolor1}{rgb}{0.00000,0.44700,0.74100}%
\definecolor{mycolor2}{rgb}{0.85000,0.32500,0.09800}%
\definecolor{mycolor3}{rgb}{0.92900,0.69400,0.12500}%
\definecolor{mycolor4}{rgb}{0.49400,0.18400,0.55600}%
\definecolor{mycolor5}{rgb}{0.46600,0.67400,0.18800}% 
\definecolor{mycolor6}{rgb}{0.30100,0.74500,0.93300}%
\definecolor{mycolor7}{rgb}{0.63500,0.07800,0.18400}%

\newcommand\minmod{\text{minmod}}  

\newcommand\T{\rule{0pt}{3ex }}       % Top table strut
\newcommand\B{\rule[-4ex]{0pt}{4ex }} % Bottom table strut

\newcommand\TM{\rule{0pt}{2.8ex }}       % Top matrix strut
\newcommand\BM{\rule[-2ex]{0pt}{2ex }} % Bottom matrix strut

\newcommand{\vecn}[1]{\boldsymbol{#1}}
\DeclareRobustCommand{\solidrule}[1][0.25cm]{\rule[0.5ex]{#1}{1.5pt}}

\DeclareRobustCommand{\dashedrule}{\mbox{%
		\solidrule[2mm]\hspace{2mm}\solidrule[2mm]}}

\DeclareRobustCommand{\tikzcircle}[1]{\tikz{\filldraw[#1] (0,0) circle (0.5ex);}}	
	
	
\DeclareRobustCommand{\squaret}[1]{\tikz{\draw[#1,thick] (0,0) rectangle (0.2cm,0.2cm);}}
\DeclareRobustCommand{\circlet}[1]{\tikz{\draw[#1,thick] (0,0) circle [radius=0.1cm];}}
\DeclareRobustCommand{\trianglet}[1]{\tikz{\draw[#1,thick] (0,0) --
		(0.25cm,0) -- (0.125cm,0.25cm) -- (0,0);}}
\DeclareRobustCommand{\crosst}[1]{\tikz{\draw[#1,thick] (0cm,0cm) --
		(0.1cm,0.1cm) -- (0cm,0.2cm) -- (0.1cm,0.1cm) -- (0.2cm,0.2cm) -- (0.1cm,0.1cm)-- (0.2cm,0cm);}}
\DeclareRobustCommand{\diamondt}[1]{\tikz{\draw[#1,thick] (0,0) --(0.1cm,0.15cm) -- (0.2cm,0cm) -- (0.1cm,-0.15cm) -- (0,0)  ;}}
\DeclareRobustCommand{\squareF}[1]{\tikz{\filldraw[#1,fill opacity= 0.3] (0,0) rectangle (0.2cm,0.2cm);}}

\begin{document}

\maketitle

\vspace{-0.3in}
\noindent
\rule{\linewidth}{0.4pt}

\section{Goal}
We are going to use the following notation for all polynomials
\[P^r_i(x) = a^{(r)}_r(x  - x_i)^r + a^{(r-1)}_r(x  - x_i)^{r-1} + \dots + a^{(0)}_r\]

Family of polynomials:
$B^0_i$, $B^1_i$,$B^2_i$,$B^3_i$,$\dots$ $B^n_i$

Such that over all the cells to match cell averages with that do not contribute to the cell averages. 

Let's assume that we start at cell $j$ and we keep adding points to the left we then would want
\[B^1_i = a^{(1)}_1(x  - x_i)^{1}  - \frac{1}{x_{j+1/2} - x_{j-1/2}}\int_{x_{j-1/2}}^{x_{j+1/2}} a^{(1)}_1(x  - x_i)^1 dx \]

So that 
\[\int_{x_{j-1/2}}^{x_{j+1/2}}B^1_i \;dx = \int_{x_{j-1/2}}^{x_{j+1/2}}a^{(1)}_1(x  - x_i)^{1}  \;dx - \frac{x_{j+1/2} - x_{j-1/2}}{x_{j+1/2} - x_{j-1/2}}\int_{x_{j-1/2}}^{x_{j+1/2}} a^{(1)}_1(x  - x_i)^1 dx   = 0 \]

Then building to the left we would have
\[B^2_i = a^{(2)}_2(x  - x_i)^{2} - c_1(x  - x_i)  - c_0 \]

Where
\[\int_{x_{j-1/2}}^{x_{j+1/2}}B^2_i \;dx = 0\]
and
\[\int_{x_{j-3/2}}^{x_{j-1/2}}B^2_i \;dx = 0\]


I have the first group of solutions for this problem (admittedly made simpler for having equally sized cells). Would be nice to have these in more general form, anyway

\[B^1_{i} = a^{(1)}_1(x  - x_i)^{1}   \]

Quadratics
\[B^2_{i-1,i} = a^{(2)}(x  - x_i)^2 + a^{(2)} \Delta x (x  - x_i) - \frac{a^{(2)} \Delta x^2}{12}   \]
\[B^2_{i,i+1} = a^{(2)}(x  - x_i)^2 - a^{(2)} \Delta x (x  - x_i) - \frac{a^{(2)} \Delta x^2}{12}   \]

Cubics
\[B^3_{i-2,i} = a^{(3)}(x  - x_i)^3 + 3a^{(3)} \Delta x (x  - x_i)^2 + \frac{7a^{(3)} \Delta x^2}{4}(x  - x_i) - \frac{a^{(3)} \Delta x^3}{4}     \]


\[B^3_{i,i+2} = a^{(3)}(x  - x_i)^3 - 3a^{(3)} \Delta x (x  - x_i)^2 + \frac{7a^{(3)} \Delta x^2}{4}(x  - x_i) + \frac{a^{(3)} \Delta x^3}{4}   \]

So we have the hieracrhy:\\
Leftwards : $B^1_{i}$, $B^2_{i-1,i}$, $B^3_{i-2,i}$ \\
Rightwards : $B^1_{i}$, $B^2_{i,i+1}$, $B^3_{i,i+2}$


\end{document} 