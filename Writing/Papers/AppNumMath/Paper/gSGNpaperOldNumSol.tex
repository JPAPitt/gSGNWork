\documentclass[10pt]{elsarticle}

  \usepackage{pgfplots}
\pgfplotsset{compat=newest}
%% the following commands are needed for some matlab2tikz features
\usetikzlibrary{plotmarks}
\usetikzlibrary{arrows.meta}
\usepgfplotslibrary{patchplots}
\usepackage{grffile}
\usepackage{amsmath}
\usepackage{lineno}


%\usepackage{fullpage}
\usepackage[top=1in, bottom=1in, left=0.8in, right=1in]{geometry}
\usepackage{multicol}
\usepackage{caption}
\usepackage{subcaption}
\usepackage{hyperref}
\usepackage{xcolor}
\usepackage{graphicx,psfrag}
\usepackage[pdf]{pstricks}

\definecolor{lightblue}{rgb}{.80,.9,1}
\newcommand{\hl}[1]
    {\par\colorbox{lightblue}{\parbox{\linewidth}{#1}}}

\newcommand{\defn}{\stackrel{\textrm{\scriptsize def}}{=}}

\setlength{\columnsep}{0.1pc}

\title{Numerical Scheme for the Generalised Serre-Green-Naghdi Model}
\author{Jordan Pitt -- \texttt{jordan.pitt@anu.edu.au}, Christopher Zoppou -- \texttt{christopher.zoppou@anu.edu.au}, Stephen Roberts -- \texttt{stephen.roberts@anu.edu.au}}

% TIME ON EVERY PAGE AS WELL AS THE FILE NAME
\usepackage{fancyhdr}
\usepackage{currfile}
\usepackage[us,12hr]{datetime} % `us' makes \today behave as usual in TeX/LaTeX
\fancypagestyle{plain}{
\fancyhf{}
\rfoot{\emph{\footnotesize \textcopyright  Serre Notes by  J. Pitt, C. Zoppou and S. Roberts.}
 \\ File Name: {\currfilename} \\ Date: {\ddmmyyyydate\today} at \currenttime}
\lfoot{Page \thepage}
\renewcommand{\headrulewidth}{0pt}}
\pagestyle{plain}

\definecolor{mycolor1}{rgb}{0.00000,0.44700,0.74100}%
\definecolor{mycolor2}{rgb}{0.85000,0.32500,0.09800}%
\definecolor{mycolor3}{rgb}{0.92900,0.69400,0.12500}%
\definecolor{mycolor4}{rgb}{0.49400,0.18400,0.55600}%
\definecolor{mycolor5}{rgb}{0.46600,0.67400,0.18800}% 
\definecolor{mycolor6}{rgb}{0.30100,0.74500,0.93300}%
\definecolor{mycolor7}{rgb}{0.63500,0.07800,0.18400}%

\newcommand\T{\rule{0pt}{3ex }}       % Top table strut
\newcommand\B{\rule[-4ex]{0pt}{4ex }} % Bottom table strut

\newcommand\TM{\rule{0pt}{2.8ex }}       % Top matrix strut
\newcommand\BM{\rule[-2ex]{0pt}{2ex }} % Bottom matrix strut

\newcommand{\vecn}[1]{\boldsymbol{#1}}
\DeclareRobustCommand{\solidrule}[1][0.25cm]{\rule[0.5ex]{#1}{1.5pt}}

\DeclareRobustCommand{\dashedrule}{\mbox{%
		\solidrule[2mm]\hspace{2mm}\solidrule[2mm]}}

\DeclareRobustCommand{\tikzcircle}[1]{\tikz{\filldraw[#1] (0,0) circle (0.5ex);}}	
	
	
\DeclareRobustCommand{\squaret}[1]{\tikz{\draw[#1,thick] (0,0) rectangle (0.2cm,0.2cm);}}
\DeclareRobustCommand{\circlet}[1]{\tikz{\draw[#1,thick] (0,0) circle [radius=0.1cm];}}
\DeclareRobustCommand{\trianglet}[1]{\tikz{\draw[#1,thick] (0,0) --
		(0.25cm,0) -- (0.125cm,0.25cm) -- (0,0);}}
\DeclareRobustCommand{\crosst}[1]{\tikz{\draw[#1,thick] (0cm,0cm) --
		(0.1cm,0.1cm) -- (0cm,0.2cm) -- (0.1cm,0.1cm) -- (0.2cm,0.2cm) -- (0.1cm,0.1cm)-- (0.2cm,0cm);}}
\DeclareRobustCommand{\diamondt}[1]{\tikz{\draw[#1,thick] (0,0) --(0.1cm,0.15cm) -- (0.2cm,0cm) -- (0.1cm,-0.15cm) -- (0,0)  ;}}
\DeclareRobustCommand{\squareF}[1]{\tikz{\filldraw[#1,fill opacity= 0.3] (0,0) rectangle (0.2cm,0.2cm);}}

\begin{document}

\maketitle

\vspace{-0.3in}
\noindent
\rule{\linewidth}{0.4pt}



%-------------------------------------------------
\section{Abstract}
%-------------------------------------------------



%-------------------------------------------------
\section{Introduction}
%-------------------------------------------------
The generalised Serre-Green-Naghdi equations (gSGN) were recently derived by \citet{Clamond-Dutykh-2018-237}. They are a family of equations that generalise the classical Serre-Green-Naghdi equations (SGN) first derived by \citet{Serre-F-1953-857} with the addition of two parameters $\beta_1$ and $\beta_2$. The equations describe the behaviour of water waves in shallow water where the typical water depth $h_0$ is much smaller than the wavelength $\lambda$ so that the shallowness parameter $\sigma = h_0/\lambda \ll 1$. The gSGN are of particular interest to the wave modelling community as their dispersion relationship well approximates the dispersion relationship of the linear wave theory \cite{Whitham-1967-399}. The choice of $\beta$ values allows for the dispersion relationship of the gSGN to be accurate up to $\mathcal{O}\left(\sigma^2\right)$ terms, $\mathcal{O}\left(\sigma^4\right)$ terms and even $\mathcal{O}\left(\sigma^6\right)$ terms \cite{Clamond-Dutykh-2018-237,Clamond-et.al-2017-245}. Therefore, these equations provide a way to study the behaviour of water waves as higher powers of $\sigma$ are retained in the wave model. 

In previous work we have developed and validated numerical methods for a conservative reformulation of the SGN \cite{Zoppou-2014,Zoppou-etal-2016,Zoppou-etal-2017,Pitt-2019}. These numerical methods have all been based on the following approach; firstly solve an auxiliary elliptic equation containing only spatial derivatives, to obtain all the primitive variables in the conservation equation. Then evolve the remaining conservation equation using a finite volume method. The benefit of the approach was the replication of the fundamental conservation properties of the SGN \cite{Pitt-2019} and the robustness of the method in the presence of steep gradients \cite{Pitt-2018-61}. The approach has been shown to produce the theoretical accuracy of the underlying methods up to third-order \cite{Zoppou-etal-2017,Pitt-2019} with the desired conservation and linear dispersion properties \cite{Pitt-2019}. The approach has also been successfully extended to include the effects of varying bathymetry and allow for the presence of dry beds \cite{Pitt-2019}.  

It was demonstrated by \citet{Clamond-Dutykh-2018-237} that the gSGN also possesses a conservation reformulation and thus the technique described above for the SGN equations can be extended to the gSGN. The success of this technique for the SGN makes this an attractive option and would allow for an explicit, robust and conservative method for the gSGN. Hence, the purpose of this paper is a description of the extension to the numerical scheme of \citet{Zoppou-etal-2017} for the recently developed gSGN \cite{Clamond-Dutykh-2018-237}. 

This paper begins by introducing the gSGN and highlighting their important properties with regards to the developed numerical scheme. The overall numerical scheme of \citet{Zoppou-etal-2017} is then described, with a straightforward second-order implementation of the method used as an example numerical method. 

The numerical method is then validated against analytic solutions of the SGN and the Shallow Water Wave Equations (SWWE), demonstrating its convergence rate and conservation properties. Additionally, forced solutions are used to validate that all terms in the gSGN are being accurately approximated to the correct order of accuracy. Forced solutions are necessary to validate the numerical method for the more general members of this family of equations, because no analytic solutions are currently known. Together these validations demonstrate the capability of the numerical scheme to produce robust and accurate numerical methods. 


%-------------------------------------------------
\section{Generalised Serre-Green-Naghdi Equations}
%-------------------------------------------------
The gSGN were derived by \citet{Clamond-Dutykh-2018-237} using a Lagrangian field theory approach. These equations generalise the SGN that describe a depth averaged approximation to the Euler equations, where $h(x,t)$ is the height of the free-surface of the water, $u(x,t)$ is the depth averaged horizontal velocity and $g$ is the acceleration due to gravity. The gSGN generalise the SGN equations by introducing two free parameters $\beta_1$ and $\beta_2$, that when fixed result in a particular member of this family of equations. In particular the family of equations captured by gSGN contains the SWWE, the SGN, a family of regularised SWWE studied by \citet{Clamond-Dutykh-2018-237} and a family of improved dispersion SGN equations studied by \citet{Clamond-et.al-2017-245}.

The gSGN describe the conservation of mass ($h$), momentum ($uh$) and energy ($\mathcal{E}$) for water waves subject to gravitational forces like so
\begin{subequations}
\begin{align}
\begin{split}
\dfrac{\partial h}{\partial t} + \dfrac{\partial (hu)}{\partial x} = 0
\label{eq:gSGNh}
\end{split}\\
\begin{split}
\dfrac{\partial (hu)}{\partial t} + \dfrac{\partial }{\partial x} \left( hu^2 + \frac{1}{2}gh^2 + \frac{1}{2} h^2 \Gamma \right)= 0
\label{eq:gSGNuh}
\end{split}\\
\begin{split}
\dfrac{\partial\left(\mathcal{E}\right)}{\partial t} +\dfrac{\partial}{\partial x}\left[hu\left(\frac{1}{2}u^2 + \dfrac{1}{4}\beta_1h^2\dfrac{\partial u}{\partial x}\dfrac{\partial u}{\partial x} + gh\left(1 + \frac{1}{4}\beta_2\dfrac{\partial h}{\partial x}\dfrac{\partial h}{\partial x} \right)   + \frac{1}{3} h\Gamma  \right) + \frac{1}{2}\beta_2 g h^3\dfrac{\partial h}{\partial x}\dfrac{\partial u}{\partial x} \right] = 0
\label{eq:gSGNE}
\end{split}
\end{align}
where
\begin{align}
\Gamma &= \beta_1h \left[\frac{\partial u}{\partial x}\frac{\partial u}{\partial x} - \frac{\partial^2 u}{\partial x \partial t} - u\frac{\partial^2 u}{\partial x^2}\right] -  \beta_2 g\left[h \frac{\partial^2 h}{\partial x^2} + \frac{1}{2} \frac{\partial h}{\partial x}\frac{\partial h}{\partial x} \right]\\
\mathcal{E} &=\frac{1}{2}hu^2 + \dfrac{1}{4}\beta_1 h^3 \dfrac{\partial u}{\partial x}\dfrac{\partial u}{\partial x} + \frac{1}{2}gh^2\left(1 + \frac{1}{2}\beta_2 \dfrac{\partial h}{\partial x} \dfrac{\partial h}{\partial x}\right).
\end{align}
\label{eq:gSGN}
\end{subequations}
When $\beta_1 = \beta_2 = 0$ these equations reduce to the SWWE, whilst when $\beta_1 = 2/3$ and $\beta_2 = 0$ the SGN are recovered. 

Equation \eqref{eq:gSGN} holds for all $\beta$ values provided the solutions are sufficiently smooth. However, for particular $\beta$ values, for example those corresponding to the SWWE, it is possible to obtain non-smooth solutions for any pair of these equations that no longer satisfy all three equations simultaneously \cite{Pu-2018-1361}. Typically, since the mass and momentum equations are solved this results in dissipation of energy around discontinuities in solutions of the SWWE. Conversely, when mass and energy equations are solved this results in dissipation of momentum and an incorrect shock speed for jump discontinuities.

Since \eqref{eq:gSGN} describes the conservation of mass, momentum and energy, when solutions are sufficiently smooth all equations hold simultaneously and the total amounts of all these quantities remain constant in time if the system is closed. This can be seen by integrating \eqref{eq:gSGN} over the domain, and observing that the temporal derivative of the spatial integrals of mass ($h$), momentum ($uh$) and energy ($\mathcal{E}$) is zero when there is no flux across the domain boundaries. This conservation property of the gSGN for $h$, $uh$ and $\mathcal{E}$ will be used to validate the numerical method and its solutions. 

\subsection{Dispersion Relation of the Linearised gSGN}
The linear dispersion properties of water wave equations have been of particular interest \cite{Clamond-et.al-2017-245,Filippini-etal-2016-381,DoCarmo-2019-125}, as the scope of wave modelling expands to include dispersive effects. Indeed, the gSGN are especially relevant due to their dispersion relation well approximating the dispersion relation given by the linear theory for water waves \cite{Whitham-1967-399}. 

To obtain the dispersion relationship of the linearised gSGN, \eqref{eq:gSGN} is first linearised by considering small waves on a mean flow depth $h_0$ and mean velocity of the flow $u_0$. The dispersion relationship of the linearised gSGN is then obtained by seeking travelling wave solutions of the form $\exp\left(i (k x - \omega t)\right)$, as was done by \citet{Zoppou-etal-2017} to obtain
\begin{equation}
\omega^\pm = u_0 k \pm k \sqrt{gh_0} \sqrt{\dfrac{\beta_2 h_0^2 k^2 + 2}{\beta_1 h_0^2 k^2 + 2} }.
\label{eq:DispRelgSGN}
\end{equation}
which provides the angular frequency $\omega$ of travelling wave solutions of the linearised gSGN equations for waves with wavenumber $k$. The dispersion relation has a positive and negative branch corresponding to the direction of these waves. This dispersion relation \eqref{eq:DispRelgSGN} is equivalent to the dispersion relation derived by \citet{Clamond-Dutykh-2018-237} for the gSGN when $u_0 = 0$. 

The dispersion relation of the gSGN approximates the dispersion relationship of water, as can be seen by comparing their power series approximations provided below
\begin{subequations}
\begin{flalign}
\omega^\pm_{\text{water}} &= u_0 k \pm \sqrt{gk \tanh\left(k h_0\right)} \notag & \\
&= \left(u_0 \pm \sqrt{gh_0}\right) k &&\pm  \frac{-1}{6}\sqrt{gh_0} h_0^2 k^3  &&\pm \frac{19}{360} \sqrt{gh_0}  h_0^4  k^5 &&+ \mathcal{O}\left(k^7\right)\\
\omega^\pm &= \left(u_0 \pm \sqrt{gh_0}\right) k &&\pm  \frac{\left(\beta_2 - \beta_1\right) }{4} \sqrt{gh_0} h_0^2 k^3  &&\pm \dfrac{\left(3 \beta_1^2 -  2 \beta_2 \beta_1 -\beta_2^2 \right) }{32} \sqrt{gh_0} h_0^4 k^5  &&+ \mathcal{O}\left(k^7\right).
\end{flalign}
\label{eq:DispWaterPower}
\end{subequations}
Which is accurate up to the $k$ term for all $\beta$ values, accurate in the $k^3$ term when $\beta_1 = \beta_2 + 2/3$ and accurate in the $k^5$ term when $\beta_1 = \beta_2 + 2/3$ and $\beta_2 = 2/15$ \cite{Clamond-et.al-2017-245}. Since $k = 2\pi / \lambda$, these power of $k$ terms have a corresponding $\sigma$ term, thus the gSGN equations have a dispersion relationship that can be accurate up to $\mathcal{O}\left(\sigma^2\right)$ terms, $\mathcal{O}\left(\sigma^4\right)$ terms and $\mathcal{O}\left(\sigma^6\right)$ terms. Since the $\beta$ values allow us to alter the accuracy of the dispersion relationship, the gSGN allow us to consistently compare the effect of the dispersion relationship on numerical solutions and thus the physical phenomena. 


From the dispersion relation \eqref{eq:DispRelgSGN}, the phase speed $v_p$ and the group speed $v_g$ can be derived as follows
\begin{subequations}
\begin{align}
v^\pm_p &= \frac{\omega^\pm}{k} = u_0 \pm  \sqrt{gh_0} \sqrt{\dfrac{\beta_2 h_0^2 k^2 + 2}{\beta_1 h_0^2 k^2 + 2} },\\
v^\pm_g &= \frac{\partial \omega^\pm }{\partial k}= u_0  \pm  \sqrt{gh_0} \sqrt{\dfrac{\beta_2 h_0^2 k^2 + 2}{ \beta_1 h_0^2 k^2 + 2} } \left[1 +  \dfrac{\beta_2 - \beta_1 }{\left(\frac{1}{2}\beta_2 h_0^2 k^2 +1\right)\left( \left( \beta_1 - \frac{1}{3}\right) h_0^2 k^2 + 1\right)}\right].
\end{align}
\label{eq:wavespeeds}
\end{subequations}


\subsubsection{Phase Speed Bounds}
Using phase speed bounds of the SGN \citet{Hank-etal-2010-2034} and \citet{Zoppou-etal-2017} applied approximate Riemann solvers such as those of \citet{Kurganov-etal-2001-707} to solve the SGN. Thus, if the gSGN can also be shown to have bounded phase speeds then similar methods can be applied to solve the gSGN. 

To demonstrate that the phase speeds are bounded, observe that when $\beta_1 \ge 0$, $\beta_2 \ge 0$ and $h_0 k \ge 0$ then
\begin{equation*}
f(h_0k) = \dfrac{\beta_2 \left(h_0 k\right)^2 + 2}{\beta_1 \left(h_0 k\right)^2 + 2},
\end{equation*}
is a monotone function over $h_0 k$. This can be seen by reformulating and taking the derivative with respect to $h_0 k$, to obtain that 
\begin{equation*}
 \dfrac{\partial \left(f(h_0k)\right)}{\partial \left(h_0 k\right)} = \left[\beta_2 - \beta_1\right] \dfrac{4\left(h_0 k\right)}{\left( \beta_1 \left(h_0 k\right)^2 + 2\right)^2}.
\end{equation*}
The derivative is greater than zero and thus $f(h_0k)$ is monotone non-decreasing if $\beta_1 \le \beta_2$. Whilst the derivative is greater than zero and thus $f(h_0k)$ is monotone non-increasing if $\beta_1 \ge  \beta_2$. Since $v^+_p = u_0 + \sqrt{gh_0 f(h_0 k)} $ and $v^-_p = u_0 - \sqrt{gh_0 f(h_0 k)}$, given the above properties of $f(h_0k)$ under the initial assumptions $v^+_p$ is monotone non-decreasing and $v^-_p$ is monotone non-increasing when $\beta_1 \le  \beta_2$. Whereas when $\beta_1 \ge  \beta_2$ then $v^+_p$ is monotone non-increasing and $v^-_p$ is monotone non-decreasing. 

In addition to the monotonicity of $v^\pm_p$, when $k \rightarrow 0$ then $v^\pm_p \rightarrow u_0 \pm \sqrt{gh_0}$, whilst as $k \rightarrow \infty$ then $v^\pm_p \rightarrow u_0 \pm \sqrt{gh_0} \sqrt{{\beta_2}/ \beta_1 }$. Therefore, $v^\pm_p$ is monotonic and bounded at the limits of the domain, and thus bounded for all $\beta$ values provided that $\beta_1 = 0$ when $\beta_2 = 0$, otherwise in the $k \rightarrow \infty$ limit the phase speed is no longer bounded. Consequently, the methods of \citet{Hank-etal-2010-2034} and \citet{Zoppou-etal-2017} for the SGN equations can be extended to the gSGN as the phase speeds are bounded.

In addition, to the phase speed bounds we also have the following chain of inequalities when ${\beta_1} \ge \beta_2$ 
\begin{equation}
u_0 -  \sqrt{gh_0} \le  v^-_p \le u_0 - \sqrt{gh_0} \sqrt{\dfrac{\beta_2}{ \beta_1}} \le u_0 \le u_0 + \sqrt{gh_0} \sqrt{\dfrac{\beta_2}{\beta_1}} \le   v^+_p  \le u_0 +   \sqrt{gh_0}.
\label{eq:wavespeedbound1}
\end{equation}
We designate this region of $\beta$ values, as `Region 1', it is characterised by either lack of dispersion when $\beta_2 =  \beta_1$ or trailing dispersive waves when $\beta_2 <  \beta_1$. Region 1 includes the SWWE and the SGN and is consistent with the behaviour of the dispersive waves given by the linear theory for water waves \cite{Whitham-1967-399}. 

When ${\beta_2} >  \beta_1 $ the inequality chain becomes
\begin{equation}
u_0 - \sqrt{gh_0} \sqrt{\dfrac{\beta_2}{ \beta_1}} \le v^-_p \le u_0 -  \sqrt{gh_0} \le  u_0 \le u_0 + \sqrt{gh_0} \le   v^+_p  \le u_0 +  \sqrt{gh_0} \sqrt{\dfrac{\beta_2}{ \beta_1}}
\label{eq:wavespeedbound2}
\end{equation}
This will be denoted as `Region 2' and it is characterised by advancing dispersive waves. Advancing dispersive waves are not observed for water waves. Since we are only interested in water waves, this paper will be restricted to studying Region 1, although the numerical scheme is valid for both regions.   

The regions and location of important members in terms of $\beta$ values is summarised in Figure \ref{Fig:WaveSpeedRegGrid}.

\begin{figure}
	\centering
	\includegraphics[width=\textwidth]{./Figures/Simulations/Comparison/3x3Grid.pdf}
	\caption{Phase speed regions of gSGN in terms of $\beta_1$ and $\beta_2$ showing important equations, their associated dispersion relationship accuracy \eqref{eq:DispWaterPower} and an example numerical solution for the dam-break problem \eqref{eqn:DB_Init} solved using their respective $\beta$ values.}
	\label{Fig:WaveSpeedRegGrid}
\end{figure}


%-------------------------------------------------
\subsection{Alternative Conservative Form of the gSGN}
%-------------------------------------------------
\citet{Clamond-Dutykh-2018-237} provided a reformulation of \eqref{eq:gSGNuh} for the gSGN, in an analogous to the way the SGN \cite{Zoppou-etal-2017,Hank-etal-2010-2034,Li-2014-169} were reformulated. The purpose of this reformulation is to remove the mixed spatial-temporal derivative in the flux term of the momentum equation, which is difficult to treat numerically. This reformulation is obtained by introducing a new conserved quantity
\begin{gather*}
G = hu - \frac{\beta_1}{2} \dfrac{\partial }{\partial x} \left ( h^3 \dfrac{\partial u}{\partial x} \right )
\end{gather*}
and thus \eqref{eq:gSGNuh} can be written in conservation law form for $G$ as
\begin{gather*}\label{eq:G_momentum}
\dfrac{\partial G }{\partial t}  + \dfrac{\partial}{\partial x} \left ( uG + \dfrac{gh^2}{2} - \beta_1 h^3\dfrac{\partial u}{\partial x}\dfrac{\partial u}{\partial x}  - \frac{1}{2} \beta_2 g h^2  \left[h\frac{\partial^2 h}{\partial x^2} + \frac{1}{2}\frac{\partial h}{\partial x}\frac{\partial h}{\partial x}\right]\right ) = 0
\end{gather*}
When $\beta_1 = 2/3$ then $G$ is the same conserved quantity introduced for the SGN \cite{Zoppou-etal-2017,Hank-etal-2010-2034,Li-2014-169}.

This reformulation, provides the gSGN in conservation law form for both $h$ and the new conserved quantity $G$. Thus the conservative gSGN are
\begin{subequations}
\begin{align}
\dfrac{\partial h}{\partial t} &+ \dfrac{\partial (uh)}{\partial x} = 0  \label{eq:gSGN_Gh} \\
\dfrac{\partial G }{\partial t} & + \dfrac{\partial}{\partial x} \left ( uG + \dfrac{gh^2}{2} - \beta_1 h^3\dfrac{\partial u}{\partial x}\dfrac{\partial u}{\partial x}  - \frac{1}{2} \beta_2 g h^2  \left[h\frac{\partial^2 h}{\partial x^2} + \frac{1}{2}\frac{\partial h}{\partial x}\frac{\partial h}{\partial x}\right]\right ) = 0.
\label{eq:gSGN_GG}
\end{align}
with
\begin{gather}\label{eq:G_divergent}
G = uh - \frac{\beta_1}{2}\dfrac{\partial }{\partial x} \left ( h^3 \dfrac{\partial u}{\partial x} \right ).
\end{gather}
\label{eq:gSGN_G}
\end{subequations}

Now that the gSGN are written in conservation law form and have a bound on the phase speeds, they can be solved numerically using the hybrid finite volume technique described by \citet{Zoppou-etal-2017} for the SGN.

\section{Numerical Scheme}
The proposed numerical scheme for the gSGN extends those previously published for the SGN \cite{Zoppou-etal-2017,Hank-etal-2010-2034} to allow any $\beta$ values and the associated additional terms. We begin the description of the numerical scheme by giving a brief outline of the finite volume method and the discretisation therein. We then provide an overview of the numerical technique and then describe the simplest second-order implementation of this scheme in detail. With previous studies \cite{Zoppou-etal-2017,Pitt-2019} showing the sufficiency and necessity of second-order methods for the SGN equations. 

\subsection{Finite Volume Method and Discretisation}
The heart of these hybrid finite volume methods for the gSGN equations is the finite volume method used to solve the equations in conservation law form. The finite volume method solves equations of the form 
\begin{equation}
\label{eqn:Conservf}
\frac{\partial q}{\partial t} + \frac{\partial f(q)}{\partial x} = 0
\end{equation}
where $q$ is a generic conserved quantity. This is the form of the gSGN equations after the reformulation \eqref{eq:gSGN_G}. In the finite volume method \eqref{eqn:Conservf} is integrated over cells of fixed width $\Delta x$ in space and over time steps of fixed length $\Delta t$.  The midpoint of the $j^{th}$ cell is given by $x_j = x_0 + j \Delta x$ while the cell edges of the $j^{th}$ cell are given by $x_{j-1/2} = x_0 + (j - \frac{1}{2}) \Delta x$ and $x_{j+1/2} = x_0 + (j + \frac{1}{2}) \Delta x$. Likewise the $n^{th}$ time step is given by $t^n = t^0 + n \Delta t$. 

Integrating \eqref{eqn:Conservf} over both space and time results in
\begin{equation}
\bar{q}^{n+1}_j = \bar{q}^{n}_j  +  \dfrac{\Delta t}{\Delta x}\left[F^n_{j+1/2} - F^n_{j-1/2}\right]
\label{eqn:ConsFVM}
\end{equation}
where 
\begin{equation*}
\bar{q}^n_j = \frac{1}{\Delta x} \int_{x_j - \Delta x / 2}^{x_j - \Delta x / 2} q(x,t^n) \; dx
\end{equation*}
is the average of $q$ over the $j^{th}$ cell at time $t^n$ and   
\begin{equation*}
F^n_{j\pm 1/2} =\frac{1}{\Delta t} \int_{t^n}^{t^{n+1}} f(q(x_{j\pm 1/2},t)) dt
\end{equation*}
is the average flux of $q$ across the cell edge from time $t^n$ to $t^{n+1}$. Therefore, if $F^n_{j\pm 1/2}$ can be approximated with the appropriate order of accuracy, then we have an explicit method for updating the cell average of the conserved quantities through time for equations in conservation law form \eqref{eqn:Conservf}. 


\subsection{Overview}
We will now describe the numerical scheme, which uses the above finite volume formulation to solve \eqref{eq:gSGN_Gh} and \eqref{eq:gSGN_GG}. When describing this broad overview it is also useful to consider collections of the point-wise values $q$ or cell averaged values $\bar{q}$ described above over the whole domain at a particular time. Thus for $q$, we define $\vecn{q}^n$ to be the vector of $q^n_j$ values and $\bar{\vecn{q}}^n$ to be the vector of $\bar{q}^n_j$ values for all cells in the domain at time $t^n$. 

The numerical scheme for the gSGN equations based on the finite volume method \eqref{eqn:ConsFVM} then proceeds as follows. Begin at a generic $n^{th}$ time step where the vectors of cell averages of the conserved quantities $\bar{\vecn{h}}^n$ and $\bar{\vecn{G}}^n$ are either known or provided as initial conditions. 
\begin{enumerate}
	\item Solve \eqref{eq:G_divergent} using $\bar{\vecn{h}}^n$ and $\bar{\vecn{G}}^n$ to obtain an approximation to $\bar{{\vecn{u}}}^n$, which can be written
	\[\mathcal{A}\left(\bar{\vecn{h}}^n,\bar{\vecn{G}}^n\right) \boldsymbol{\rightarrow} \bar{{\vecn{u}}}^n.\]
	\item Solve \eqref{eq:gSGN_Gh} and \eqref{eq:gSGN_GG}, using the finite volume method \eqref{eqn:ConsFVM} with an approximate Riemann solver such as the one described by \citet{Kurganov-etal-2001-707}, to obtain $\bar{h}^{n+1}_j$ and $\bar{G}^{n+1}_j$ at the next time step, like so
	\[\mathcal{F}\left(\bar{\vecn{h}}^n,\bar{\vecn{G}}^n,\bar{{\vecn{u}}}^n\right) \boldsymbol{\rightarrow}\bar{\vecn{h}}^{n+1 },\bar{\vecn{G}}^{n+1}.\]
	\item Since $\mathcal{F}$ is only first order accurate in time, Steps 1 and 2 are repeated a sufficient amount of times and the appropriate order approximations to $\bar{\vecn{h}}^{n+1 }$,$\bar{\vecn{G}}^{n+1}$ are obtained using a convex combination of these repetitions of Steps 1 and 2 to obtain a Strong Stability Preserving (SSP) Runge Kutta time stepping method \cite{Gottlieb-etal-2003-89}. 
\end{enumerate}

This numerical scheme produces the numerical methods of \citet{Hank-etal-2010-2034}, \citet{Zoppou-etal-2017} and \citet{Pitt-2019} when $\beta_1 = 2/3$ and $\beta_2 = 0$. Additionally, when $\beta_1 = \beta_2 = 0$ the gSGN reduces to the SWWE, and this numerical scheme reduces to a finite volume method \cite{Roberts-2003-129}.  

\subsection{Example Implementation}
To demonstrate the utility of this numerical scheme we present a simple second-order implementation, and then validate it in the following section. The description of this numerical method will be broken up into the steps described in Section 4.2 for simplicity and also to highlight the interchangeability of the different parts of the scheme.

\subsubsection{Step 1 - Solution of Elliptic Equation}
To solve \eqref{eq:G_divergent} with $\bar{\vecn{h}}^n$ and $\bar{\vecn{G}}^n$ to obtain $\bar{\vecn{u}}^n$ we use the observation that the cell average and the cell nodal value are equivalent up to second-order accuracy so that $\bar{\vecn{q}}^n = {\vecn{q}}^n + \mathcal{O}\left(\Delta x^2\right)$ for all the quantities of interest. Assuming that $u$ is sufficiently smooth, then a second-order central finite difference approximation can be used to accurately solve \eqref{eq:G_divergent}. 

The second-order central finite difference approximation to \eqref{eq:G_divergent} can be written as
\begin{equation}
{\vecn{u}}^n = \vecn{A}^{-1} {\vecn{G}}^n
\label{eq:FDGeqn}
\end{equation}
where $\vecn{A}$ is a tri-diagonal matrix with the sub-diagonal, diagonal and super-diagonal given by
\begin{align*}
A_{j,j-1} &=  -\frac{\beta_1}{2}  \left[  \dfrac{\left(h_j^n\right)^3}{\Delta x^2} -  \dfrac{3\left(h_j^n\right)^2}{2\Delta x} \dfrac{h_{j+1}^n - h_{j-1}^n}{2\Delta x}\right] \\
A_{j,j} &= h^n_j + \beta_1\dfrac{\left(h_j^n\right)^3}{\Delta x^2} \\
A_{j,j+1} &=  -\frac{\beta_1}{2}  \left[  \dfrac{\left(h_j^n\right)^3}{\Delta x^2} +  \dfrac{3\left(h_j^n\right)^2}{2\Delta x} \dfrac{h_{j+1}^n - h_{j-1}^n}{2\Delta x}\right]
\end{align*}
with all other elements in $\vecn{A}$ being zero. 

The finite difference approximation \eqref{eq:FDGeqn} can be solved with any matrix solver, for this method we use the explicit Thomas Algorithm \cite{Conte-DeBoor-1980}. This is an efficient method provided $\vecn{A}$ is non-singular which is the case as long as $h_{j}^n > 0$, which is true for all the problems of interest in this paper. 

Thus as desired we have 
\begin{equation}
\mathcal{A}\left(\bar{\vecn{h}}^n,\bar{\vecn{G}}^n\right) = \vecn{A}^{-1} {\vecn{G}}^n = \vecn{u}^n \boldsymbol{\rightarrow} \bar{{\vecn{u}}}^n. 
\label{eq:A_secondord}
\end{equation}


\subsubsection{Step 2 -  Finite Volume Method  - Reconstruction}
The finite volume method of \citet{Kurganov-etal-2001-707} relies on approximations of all the terms in the spatial derivative of the flux at the cell edges $x_{j\pm1/2}$. Thus the following quantities require second-order approximations at the cell edges: $u$, $h$, $G$, $\partial u / \partial x$, $\partial h / \partial x$, $\partial^2 h / \partial x^2$. In this paper we are interested in reproducing solutions to either smooth analytic and forced solutions or the discontinuous dam-break solutions using the SWWE where only $h$, $u$ and $G = uh$ need to be approximated. For this reason our approximations to  $u$, $h$, $G$, will allow for discontinuities and thus use limiting while the approximations to the derivatives $\partial u / \partial x$, $\partial h / \partial x$, $\partial^2 h / \partial x^2$ will assume that these quantities are smooth and thus do not require limiting.

The reconstructions for all these quantities can be summarised for a general quantity $q$ like so
\begin{align}
q^+_{j-1/2} = \bar{q}_j - \frac{\Delta x}{2} s_j & & \text{and} & &
q^-_{j+1/2} = \bar{q}_j + \frac{\Delta x}{2} s_j
\end{align}
where
\begin{equation}
s_j = \text{minmod}\left(\theta \dfrac{\bar{q}_j - \bar{q}_{j-1}}{\Delta x},  \dfrac{\bar{q}_{j+1} - \bar{q}_{j-1}}{2\Delta x},\theta \dfrac{\bar{q}_{j+1} - \bar{q}_{j}}{\Delta x}\right).
\end{equation}
The minmod function is defined as follows
\begin{equation}
\text{minmod}\left(a,b,c\right) = \left\lbrace \begin{array}{l c l}
\min{\left(a,b,c\right)} & \text{when} & a<0, b<0, c<0 \\
\max{\left(a,b,c\right)} & \text{when} & a>0, b>0, c>0 \\
0 & & \text{otherwise}
\end{array} \right. . 
\end{equation}
This gives a reconstruction at the cell level for each cell edge, so cell $j$ gives rise to $q^+_{j-1/2}$ and $q^-_{j+1/2} $ while cell $j+1$ gives rise to $q^+_{j+1/2}$ and $q^-_{j+3/2}$, all of which are required in the flux approximation.

For $\partial u / \partial x$, $\partial h / \partial x$ and $\partial^2 h / \partial x^2$ we assume that the quantities are smooth, and so using the appropriate order finite difference approximation at the cell edges is sufficient. We demonstrate this for cell $x_{j+1/2}$ and a general quantity $q$

\begin{align*}
\left[\dfrac{\partial q}{\partial x} \right]_{j+1/2} &= \dfrac{q_{j+1} - q_j}{\Delta x}, \\
\left[\dfrac{\partial^2 q}{\partial x^2} \right]_{j+1/2} &=  \dfrac{q_{j+2} - q_{j+1} - q_j + q_{j-1}}{2 \Delta x^2} .
\end{align*}


\subsubsection{Step 2 -  Finite Volume Method - Flux Approximation}
To solve \eqref{eq:gSGN_Gh} and \eqref{eq:gSGN_GG} which are both in conservation law form \eqref{eqn:Conservf} we use \eqref{eqn:ConsFVM}. Since we begin with $\bar{\vecn{h}}^n$ and $\bar{\vecn{G}}^n$, we only require an approximation to $F^n_{j\pm1/2}$ to obtain $\bar{\vecn{h}}^{n+1}$ and $\bar{\vecn{G}}^{n+1}$.

In this numerical scheme the approximate Riemann solver described by \citet{Kurganov-etal-2001-707} is used to calculate $F^n_{j\pm1/2}$. The major advantage of this scheme is that it only requires bounds on the phase speeds. Only the calculation of the flux term $F^n_{j+1/2}$ is demonstrated as the process to calculate the flux term $F^n_{j-1/2}$ is identical but with different cells. For $q$ the flux is approximated by
\begin{equation}\label{eqn:HLL_flux}
F^n_{j+\frac{1}{2}} = \dfrac{a^+_{j+\frac{1}{2}} f\left(q^-_{j+\frac{1}{2}}\right) - a^-_{j+\frac{1}{2}} f\left(q^+_{j+\frac{1}{2}}\right)}{a^+_{j+\frac{1}{2}} - a^-_{j+\frac{1}{2}}}  + \dfrac{a^+_{j+\frac{1}{2}} \, a^-_{j+\frac{1}{2}}}{a^+_{j+\frac{1}{2}} - a^-_{j+\frac{1}{2}}} \left [ q^+_{j+\frac{1}{2}} - q^-_{j+\frac{1}{2}} \right ]
\end{equation}

where $a^+_{j+\frac{1}{2}}$ and $a^-_{j+\frac{1}{2}}$ are given by the phase speed bounds and all quantities on the right hand side are computed at time $t^n$. Applying the phase speed bounds \eqref{eq:wavespeedbound1} and \eqref{eq:wavespeedbound2} we obtain
\begin{align}
a^-_{j+\frac{1}{2}} &= \min\left\lbrace 0\;,\;  u^-_{j + 1/2} - \max{\left(1 , \sqrt{\frac{\beta_2}{\beta_1}}\right)} \sqrt{g h^-_{j + 1/2}}  \;,\;u^+_{j + 1/2} -\max{\left(1 , \sqrt{\frac{\beta_2}{\beta_1}}\right)} \sqrt{g h^+_{j + 1/2}} \right\rbrace  ,\\
a^+_{j+\frac{1}{2}} &= \max\left\lbrace 0 \;,\;  u^-_{j + 1/2} + \max{\left(1 , \sqrt{\frac{\beta_2}{\beta_1}}\right)}\sqrt{g h^-_{j + 1/2}}  \;,\;u^+_{j + 1/2} + \max{\left(1 , \sqrt{\frac{\beta_2}{\beta_1}}\right)}\sqrt{g h^+_{j + 1/2}} \right\rbrace
\label{eqn:WaveSpeedBoundsFluxApprox}
\end{align}
where the $\max{\left(1 , \sqrt{{\beta_2}/{\beta_1}}\right)}$ accounts for the different phase speed bounds  \eqref{eq:wavespeedbound1} and \eqref{eq:wavespeedbound2}, which depend on the choice of $\beta$ values. 

The flux functions $f(q^-_{j+\frac{1}{2}})$ and $f(q^+_{j+\frac{1}{2}})$ are evaluated using the reconstructed values of the $j^{th}$ and $(j+1)^{th}$ cell respectively. From the continuity equation \eqref{eq:gSGN_Gh} we have
\begin{align*}
f\left(h^\pm_{j+\frac{1}{2}}\right) &= u^\pm_{j + 1/2}  h^\pm_{j + 1/2}.
\end{align*}

For the evolution of $G$ equation \eqref{eq:gSGN_GG} we have 
\begin{multline}
f\left(G^\pm_{j+\frac{1}{2}}\right) =  u^\pm_{j + 1/2} G^\pm_{j + 1/2}  + \frac{g}{2}\left(h^\pm_{j + 1/2} \right)^2 - \beta_1\left(h^\pm_{j + 1/2}\right)^3 \left(\left[\frac{\partial {u}}{\partial x} \right]_{j + 1/2} \right)^2 \\ - \frac{1}{2} \beta_2 g \left(h^\pm_{j + 1/2}\right)^2  \left[h^\pm_{j + 1/2}\left[\dfrac{\partial^2 h}{\partial x^2} \right]_{j+1/2} + \frac{1}{2}\left(\left[\dfrac{\partial h}{\partial x} \right]_{j+1/2}\right)^2\right].
\end{multline}
where the derivatives are assumed to be smooth across the cell edge, and thus do not require different superscripts.

Since the reconstructions were given above for all these quantities, we can approximate $F^n_{j\pm1/2}$ using \eqref{eqn:HLL_flux} and thus employ \eqref{eqn:ConsFVM} to obtain $\bar{\vecn{h}}^{n+1}$ and $\bar{\vecn{G}}^{n+1}$ resulting in
\begin{equation}
\mathcal{F}\left(\bar{\vecn{h}}^n,\bar{\vecn{G}}^n,\bar{{\vecn{u}}}^n\right) =   \bar{q}^{n}_j  +  \dfrac{\Delta t}{\Delta x}\left[F^n_{j+1/2} - F^n_{j-1/2}\right] \boldsymbol{\rightarrow} \bar{\vecn{h}}^{n+1 },\bar{\vecn{G}}^{n+1}
\label{eq:F_secondord}
\end{equation}
as desired. 

\subsubsection{Step 3 - Runge-Kutta Time Stepping}
Combining both Steps 1 and Steps 2, provides a spatially second-order scheme with first-order time stepping. To arrive at a fully second-order method, we repeat Steps 1 and Steps 2 according to the second-order SSP Runge Kutta method \cite{Gottlieb-etal-2003-89}. Doing this we obtain the following scheme making use of the above implementation of $\mathcal{A}$, \eqref{eq:A_secondord} and $\mathcal{F}$, \eqref{eq:F_secondord}
\begin{align*}
\bar{\vecn{h}}^{(1)},\bar{\vecn{G}}^{(1)} &= \mathcal{F}\left(\bar{\vecn{h}}^n,\bar{\vecn{G}}^n,\mathcal{A}\left(\bar{\vecn{h}}^n,\bar{\vecn{G}}^n\right) \right) \\
\bar{\vecn{h}}^{(2)},\bar{\vecn{G}}^{(2)} &= \mathcal{F}\left(\bar{\vecn{h}}^{(1)},\bar{\vecn{G}}^{(1)},\mathcal{A}\left(\bar{\vecn{h}}^{(1)},\bar{\vecn{G}}^{(1)}\right) \right)\\
\bar{\vecn{h}}^{n+1},\bar{\vecn{G}}^{n+1} &= \frac{1}{2}\left(\bar{\vecn{h}}^n + \bar{\vecn{h}}^{(2)} \right) ,  \frac{1}{2}\left(\bar{\vecn{G}}^n + \bar{\vecn{G}}^{(2)} \right). 
\end{align*}
obtaining a stable fully second-order method for solving \eqref{eq:gSGN_G}. 

\subsection{Boundary Conditions}
For the purposes of the validation below and for simplicity, we have applied Dirichlet boundary conditions using ghost cells over which the values of $h$, $G$ and $u$ are known. Thus in addition to the $m$ cells inside the boundary we have an additional $l$ cells either side of it so that we have the left ghost cells $-l$, $-l + 1$, $\dots$, $-1$ and the right ghost cells $m$, $m+1$, $\dots$, $m + l -1$. Since the numerical method has a maximum stencil that extends $2$ cells beyond the target cell, then $l$ must be at least $2$. 

This boundary condition is applied to Step 1 by extending the matrix equation \eqref{eq:FDGeqn} to vectors containing the ghost cells $\vecn{\hat{u}}^n$ and $\vecn{\hat{G}}^n$ like so
\begin{align}
\vecn{\hat{u}}^n &= \left[u^n_{-l} \; \dots \; u^n_{-1} \; u^n_{0}\; \dots \; u^n_{m-1}\; u^n_{m} \; \dots \; u^n_{m + l -1}   \right]^T \\
\vecn{\hat{G}}^n &= \left[G^n_{-l} \; \dots \; G^n_{-1} \; G^n_{0}\; \dots \; G^n_{m-1}\; G^n_{m} \; \dots \; G^n_{m + l -1}   \right]^T.
\end{align}

Likewise the associated extended matrix $\vecn{\hat{A}}$ in \eqref{eq:FDGeqn} is the same as $\vecn{A}$ when $0 \le j \le m -1$ defined above with the additional associated ghost cell values given by
\begin{equation}
\hat{A}_{j,j-1} =  0  \quad \quad
\hat{A}_{j,j} = 1 \quad \quad
\hat{A}_{j,j+1} = 0
\end{equation}
when $j < 0$ or $j > m -1$. 

For Step 2 there is no need to reconstruct the quantities inside the ghost cells because $h$, $G$ and $u$ are given. 


\subsection{Courant-Frederichs-Lewy Condition}
To ensure the stability of the finite volume method \eqref{eqn:ConsFVM} the Courant-Friedrichs-Lewy (CFL) condition \cite{Courant-etal-1967-215} is used. The CFL condition is necessary for stability and ensures that time steps are small enough so that information is only transferred between neighbouring cells. For the gSGN equations the CFL condition is 
\begin{equation}
\Delta t \le \frac{Cr }{\max_{j} \left\lbrace a^\pm_{j+1/2} \right\rbrace} \Delta x
\label{eqn:CFLcond}
\end{equation}
where $a^\pm_{j+1/2} $ are the phase speed bounds used in the flux approximation \eqref{eqn:WaveSpeedBoundsFluxApprox} and $0\le Cr \le 1$ is the Courant number. 

\section{Validation}
The example numerical method described above is validated using analytic solutions for particular $\beta$ values that correspond to the SGN and the SWWE as well as a forced solution. Together these tests demonstrate the ability of the method to reproduce analytic solutions to important members of the gSGN family, as well as assessing the accuracy of the numerical method's approximation to all terms of the gSGN.

\subsection{Convergence and Conservation Measures}
To validate the produced numerical solutions we make use of measures of convergence and conservation. The measure of convergence will be the relative distance between the numerical solution and the equivalent analytic or forced solution using the $L_2$ norm. While the conservation properties of the numerical method will be measured by numerically approximating the energy in the initial conditions and the numerical solution and comparing them using a measure called $C_1$.

For a quantity $q$ with a vector of its analytic or forced solution at the cell midpoints $\vecn{q}$ and the numerical solution at the cell midpoints $\vecn{q}^*$, the discrete non-dimensional $L_2$ norm is
\begin{equation}
\label{eqn:Conv_Error}
L_2\left(\vecn{q},\vecn{q}^*\right) = \sqrt{ \dfrac{\sum_{j = 0}  \left[q_j^2 - \left(q^*_j\right)^2 \right]}{\sum_{j = 0}  \left[q_j^2 \right]}}
\end{equation}
where the time-step superscripts were suppressed for simplicity.

For a quantity $q$ with a vector of its values at the $n^{th}$ time step $\vecn{q}^n$, the total amount of the quantity is approximated by $C(\vecn{q}^n)$. The method for this is the same as the method described by \citet{Zoppou-etal-2017}, which has a higher order of accuracy than the numerical method for the gSGN. Using the numerical approximation to the total amounts, the conservation error is obtained using
\begin{equation}
\label{eqn:Cons_Error}
C_1\left(\vecn{q}^0,\vecn{q}^n\right) = \left \lbrace \begin{array}{l c r}
\dfrac{\left | C(\vecn{q}^0) - C(\vecn{q}^n) \right |}{\left| C(\vecn{q}^0) \right|} &,& \left| C(\vecn{q}^0) \right| > 0 \\ \\
\left | C(\vecn{q}^0) - C(\vecn{q}^n) \right |&,& \left| C(\vecn{q}^0) \right| = 0 
\end{array} \right. .
\end{equation}

%\begin{itemize}
%	\item Analytic solutions - we recover them (conservation and norm)
%	\item Forced solutions - our numerical method can handle any combination of beta values, all terms are approximated with correct order of accuracy. Limiters on gradients off. 
%\end{itemize}

\subsection{Analytic Solutions}
The analytic solutions used to validate the numerical method, are the solitary travelling wave solution of the SGN and the dam-break solution of the SWWE. The solitary travelling wave solution is a smooth travelling wave solution, that assesses the balance of the non-linear and dispersive terms in the gSGN. Whereas the dam-break solution of the SWWE demonstrates the robustness of the method in the presence of steep gradients. 

\subsubsection{SGN ($\beta_1 = 2/3$ and $\beta_2 = 0$) - Solitary Travelling Wave Solution}
When $\beta_1 = 2/3$ and $\beta_2 = 0$ the gSGN are equivalent to the SGN which admit the following travelling wave solution \cite{El-etal-2006}
\begin{subequations}
	\begin{equation}
	h(x,t) = a_0 + a_1 \text{sech}^2\left( \kappa (x - ct) \right),
	\end{equation}
	\begin{equation}
	u(x,t) = c \left( 1- \dfrac{a_0}{h(x,t)} \right),
	\end{equation}
	where
	\begin{equation}
	\kappa = \dfrac{\sqrt{3a_1}}{2a_0 \sqrt{a_0 + a_1}}
	\end{equation}
	and
	\begin{equation}
	c = \sqrt{g\left(a_0 + a_1\right)}
	\label{eq:Sol_speed}
	\end{equation}
\end{subequations}
is the speed of the wave.

This travelling wave solution is maintained due to a balance between the dispersive terms and the non-linear terms in the momentum equation \eqref{eq:gSGNuh}. This balance results in a solitary wave that is advected with a constant speed without a change in shape. Validating the numerical solutions for the gSGN solver using this solution tests the balance between these terms in \eqref{eq:gSGNuh}, and allows us to verify the method's conservation of energy as the solution is smooth. To enable a comparison between the numerical method and the SGN solver of \citet{Zoppou-etal-2017} and \citet{Pitt-2019} the chosen solitary travelling wave parameters were $a_0 = 1m$ and $a_1 = 0.7m$ and the acceleration due to gravity at the earth surface, $g = 9.81 m^2/s$ was used.

The numerical solution was solved over the domain $\left[-200m,200m\right]$ from $t=0s$ until $t=30s$. To ensure that the SGN solution is recovered, $\beta_1 = 2/3$ and $\beta_2 = 0$. The spatial resolution was varied like so $\Delta x = 400 / (100 \times 2^{l})$, where $l$ was increased from $0$ to $12$. To satisfy the CFL condition \eqref{eqn:CFLcond} the time step width $\Delta t = \Delta x  / ( 2 \sqrt{g(a_0 + a_1)})$ \cite{Pitt-2019} was chosen. The limiting parameter $\theta = 1.2$, was chosen to match previous numerical experiments \cite{Zoppou-etal-2017,Pitt-2019}.

Example numerical solutions for $h$, $u$ and $G$ with $\Delta x = 400 / (100 \times 2^{6}) \approx 0.06m$ are plotted in Figure \ref{Fig:Sol_Ex}. These examples demonstrate that the numerical solutions can reproduce the analytic solutions well.
%
\begin{figure}
	\centering
	\begin{subfigure}{0.32\textwidth}
		\centering
		\includegraphics[width=\textwidth]{./Figures/Simulations/Validation/Serre/hEx.pdf}
		\caption{$h$}
	\end{subfigure}
	\begin{subfigure}{0.32\textwidth}
		\centering
		\includegraphics[width=\textwidth]{./Figures/Simulations/Validation/Serre/GEx.pdf}
		\caption{$G$}
	\end{subfigure}
	\begin{subfigure}{0.32\textwidth}
		\centering
		\includegraphics[width=\textwidth]{./Figures/Simulations/Validation/Serre/uEx.pdf}
		\caption{$u$}
	\end{subfigure}
	\caption{Plot of comparing initial (\solidrule), analytic solution ({\color{blue}\solidrule}), and numerical solution with $\Delta x \approx 0.06m$ (\tikzcircle{red}) at $t = 30s$.}
	\label{Fig:Sol_Ex}
\end{figure}

The convergence properties of the method as $\Delta x$ varies are given in Figure \ref{Fig:Sol_Comp}
where Figure \ref{Fig:Sol_Comp_Conv} shows the $L_2$ norm calculated over the whole domain and Figure \ref{Fig:Sol_Comp_Peak} shows the $L_2$ norm at the peak only. The $L_2$ norm over the whole domain demonstrates that all quantities are approximated with second-order accuracy, and thus the method is second-order accurate. The $L_2$ norm at the peak was measured by taking the relative difference between the numerical solutions value at the peak of the wave which are given by $\max\left({\vecn{h}^n}\right)$, $\max\left({\vecn{u}^n}\right)$ to the analytic values which are $a_0 + a_1$ and $c a_0 / \left(a_0 + a_1\right)$ respectively. The $L_2$ norm at the peak demonstrates that even with the slope limiting and the diffusion introduced by the flux approximation \cite{Pitt-2018-61}, the numerical method introduces little diffusion and is second-order accurate. 
%
\begin{figure}
	\centering
	\begin{subfigure}{0.49\textwidth}
		\centering
		\includegraphics[width=\textwidth]{./Figures/Simulations/Validation/Serre/NormResults.pdf}
		\caption{Whole Domain}
		\label{Fig:Sol_Comp_Conv}
	\end{subfigure}
	\begin{subfigure}{0.49\textwidth}
		\centering
		\includegraphics[width=\textwidth]{./Figures/Simulations/Validation/Serre/PeakError.pdf}
		\caption{Peak Only}
		\label{Fig:Sol_Comp_Peak}
	\end{subfigure}
	\caption{Convergence plots for $h$ (\squaret{blue}) , $G$ (\circlet{red}) and $u$ (\trianglet{black}) as $\Delta x$ varies.}
	\label{Fig:Sol_Comp}
\end{figure}

The conservation error $C_1$ \eqref{eqn:Cons_Error} of the numerical solutions as $\Delta x$ varies is given in Figure \ref{Fig:Sol_Comp_Cons}. This conservation measure demonstrates that the finite volume method conserves $h$ and $G$ up to round-off error, which accumulates as $\Delta x$ increases due to the additional number of calculations. Since $uh$ and $\mathcal{E}$ are not the conserved quantities in our finite volume based method, they are not conserved up to round-off error. The conservation properties of $uh$ and $\mathcal{E}$ are better than expected, with both exhibiting conservation errors that have better than second-order convergence in $\Delta x$. 
%
\begin{figure}
	\centering
	\includegraphics[width=0.49\textwidth]{./Figures/Simulations/Validation/Serre/EnergyResults.pdf}
	\caption{Conservation plot for $h$ (\squaret{blue}) , $G$ (\circlet{red}), $uh$ (\trianglet{black}) and $\mathcal{E}$ (\crosst{green!80!black}) as $\Delta x$ varies.}
	\label{Fig:Sol_Comp_Cons}
\end{figure}

By locating the cell on which $\vecn{h}^n$ achieves its maximum, we can provide an upper and lower bound for the speed of the numerical solution to the travelling wave problem where the lower bound is $c^*_{-}= (x_{\text{peak cell}} - 0.5 \Delta x) / t$ and the upper bound is $c^*_{+} = (x_{\text{peak cell}} + 0.5 \Delta x) / t$. These upper and lower bounds are compared to the analytic value $c$ given by \eqref{eq:Sol_speed} in Figure \ref{Fig:Sol_Comp_Speed}. This figure demonstrates the diffusion of the numerical scheme, as when $\Delta x$ is large, both the upper and lower bounds on the wave speed are below the analytic value. However, this diffusion becomes negligible when $\Delta x$ is small, and for the lowest $\Delta x$ value we observe that $ c^*_{-}< c  < c^*_{+}$, indicating that the peak is travelling at the correct speed up to cell width accuracy.
% 
\begin{figure}
	\centering
	\includegraphics[width=0.49\textwidth]{./Figures/Simulations/Validation/Serre/PeakSpeedEstimates.pdf}
	\caption{Plot of $c^*_{+} / c $ (\crosst{blue}),  $c^*_{-} / c$ (\crosst{red}) and the analytic value (\dashedrule) as $\Delta x$ varies.}
	\label{Fig:Sol_Comp_Speed}
\end{figure}

These results agree well with the numerical solutions of \citet{Pitt-2019}, who compared various numerical methods for the SGN.

\subsubsection{SWWE ( $\beta_1= \beta_2 =0$ ) - Dam-break Solution }
When $\beta_1=\beta_2 =0$ the gSGN equations reduce to the SWWE, and consequently $G = uh$. The SWWE possess an analytic solution to the dam-break problem given by the initial conditions
\begin{subequations}
\begin{align}
h(x,0) & = \left\lbrace \begin{array}{c c}
h_0 & x < 0\\
h_1 & x \ge 0
\end{array} \right.  \\
u(x,0) &= 0 \\
G(x,0) &= 0.
\end{align}
\label{eqn:DB_Init}
\end{subequations}

The solution to the dam-break problem using the conservation of mass and momentum equations is given by
\begin{subequations}
\begin{align}
h(x,t) &= \left \lbrace \begin{array}{l c r}
h_0 &,& x \le -t\sqrt{g h_0} \\
\frac{4}{9g} \left(\sqrt{gh_0} - \frac{x}{2t}\right)^2 &,&  -t\sqrt{g h_0} < x \le t \left(u_2 - \sqrt{g h_2}\right)  \\
h_2 &,&  t \left(u_2 - \sqrt{g h_2}\right) < x \le t S_2  \\
h_1 &,&   t S_2 \le x \\
\end{array} \right.  \\
u(x,t) &= \left \lbrace \begin{array}{l c r}
0 &,& x \le -t\sqrt{g h_0} \\
\frac{2}{3} \left(\sqrt{gh_0} + \frac{x}{t}\right) &,&  -t\sqrt{g h_0} < x \le t \left(u_2 - \sqrt{g h_2}\right)  \\
u_2 &,&  t \left(u_2 - \sqrt{g h_2}\right) < x \le t S_2  \\
0 &,&   t S_2 \le x \\
\end{array} \right. .
\end{align}
\end{subequations}
%
The constant state values $h_2$ and $u_2$ and the shock speed $S_2$ can be calculated for any initial conditions by solving
\begin{subequations}
\begin{align}
\label{eq:SWWEMiddleState}
h_2 &= \dfrac{h_0}{2} \left(  \sqrt{1 + 8 \left( \dfrac{2 h_2}{h_2 - h_0} \left(\dfrac{\sqrt{gh_1} - \sqrt{gh_2}}{\sqrt{gh_0}}\right)\right)^2 } - 1 \right) \\
u_2 &= 2\left(\sqrt{gh_1} - \sqrt{gh_2} \right),\\
S_2 &= \dfrac{2 h_2}{h_2 - h_1}\left(\sqrt{gh_0} - \sqrt{gh_2} \right). \label{eq:SWWEMiddleState_S2}
\end{align}
\end{subequations}

The initial conditions as well as the analytic solution are discontinuous. Due to the discontinuities the solutions to the initial conditions are not unique, as solving any pair of the 3 conservation equations \eqref{eq:gSGN}, gives solutions with similar structures but different shock speeds. The solution presented above is solution of the mass and momentum equations, as these equations are the basis of the numerical method.

A number of numerical experiments were run for the dam-break problem with $h_0 = 2m$ and $h_1 = 1m$. The domain of the solution was $\left[-250m,250m\right]$ with a final time of $t=35s$.  The spatial resolution was varied like so $\Delta x = 500 / (100 \times 2^{l})$ where $l$ was increased from $0$ to $12$. To satisfy the CFL condition \eqref{eqn:CFLcond} the time step length $\Delta t = \Delta x  / \left( 2 \sqrt{g h_0}\right)$ was used. The limiting parameter $\theta = 1.0$ and the acceleration due to gravity $g = 9.81 m^2/s$ were used. 

Example numerical solutions with the spatial resolution $\Delta x = 500 / (100 \times 2^{5}) \approx  0.15m$ and the analytic solutions for $h$, $G$ and $u$ at the final time are plotted in Figure \ref{Fig:DB_Ex}. These figures demonstrate that the method is robust in the presence of steep gradients and accurately reproduces the analytic solution. 
%
\begin{figure}
	\centering
	\begin{subfigure}{0.32\textwidth}
		\centering
		\includegraphics[width=\textwidth]{./Figures/Simulations/Validation/DBSWWE/hEx.pdf}
		\caption{$h$}
	\end{subfigure}
	\begin{subfigure}{0.32\textwidth}
		\centering
		\includegraphics[width=\textwidth]{./Figures/Simulations/Validation/DBSWWE/GEx.pdf}
		\caption{$G = uh$}
	\end{subfigure}
	\begin{subfigure}{0.32\textwidth}
		\centering
		\includegraphics[width=\textwidth]{./Figures/Simulations/Validation/DBSWWE/uEx.pdf}
		\caption{$u$}
	\end{subfigure}
	\caption{Comparison of initial (\solidrule), analytic solution ({\color{blue}\solidrule}), and numerical solution with $\Delta x \approx 0.15m$ (\tikzcircle{red}) at  $t=35s$.}
	\label{Fig:DB_Ex}
\end{figure}

The presence of discontinuities in the analytic solution, make accurately assessing convergence of the numerical solutions difficult. To circumvent these issues, the convergence measure has been restricted to comparing the numerical and analytic solutions for the constant region between the rarefaction fan and the shock. This modified convergence measure as $\Delta x$ varies is plotted in Figure \ref{Fig:DB_Comp_Conv}. This figure demonstrates that the scheme retains it's second-order accuracy away from discontinuities, as desired. 

Since the analytic solution contains discontinuities all three conservation laws for $h$, $G$ and $\mathcal{E}$ are not all satisfied simultaneously \cite{Pu-2018-1361}. Since we solved equations for $h$ and $G$, these quantities are conserved in the analytic solutions however, $\mathcal{E}$ is no longer conserved and energy is lost as the shock propagates. I have taken account of the energy dissipation of the analytic solution by introducing a new measure $\mathcal{E}^*$ which measures the conservation error of the energy by comparing the total energy in the numerical solution to total energy in the analytic solution at the final time rather than the total energy in the initial conditions. The conservation error for mass, momentum and energy as calculated normally and the new dissipation corrected energy are all compared in Figure \ref{Fig:DB_Comp_Cons}. This figure demonstrates that due to the use of the finite volume method even in the presence of discontinuities the conserved quantities $h$ and $G$ are conserved up to round-off error, leading to their conservation error increasing as $\Delta x$ increases. Since $G = uh$, this means that $uh$ is also conserved up to round-off error. Energy is not conserved by the analytic solution, and this can be seen as the conservation error of $\mathcal{E}$ does not improve as $\Delta x$ decreases. However, when accounting for the lack of energy conservation as done with $\mathcal{E}^*$ then we are able to recover first-order accuracy. Given that this solution contains a discontinuity these are good conservation results. 
%
\begin{figure}
	\centering
	\begin{subfigure}{0.49\textwidth}
		\centering
		\includegraphics[width=\textwidth]{./Figures/Simulations/Validation/DBSWWE/ConvergenceInConstantState.pdf}
		\caption{$L_2$ for constant state for $h$ (\squaret{blue}) , $G$ (\circlet{red}) and $u$ (\trianglet{black})}
		\label{Fig:DB_Comp_Conv}
	\end{subfigure}
	\begin{subfigure}{0.49\textwidth}
		\centering
	\includegraphics[width=\textwidth]{./Figures/Simulations/Validation/DBSWWE/EnergyResults.pdf}
\caption{$C_1$ for $h$ (\squaret{blue}) , $G$ (\circlet{red}), $\mathcal{E}$ (\crosst{green!80!black}) and $\mathcal{E}^*$ (\diamondt{green!80!black})}
		\label{Fig:DB_Comp_Cons}
	\end{subfigure}
	\caption{Convergence and conservation plots.}
	\label{Fig:DB_Comp}
\end{figure}

To further justify the ability of the method to resolve the discontinuous analytic solution, I have found lower and upper bounds for the shock speed in the numerical solution. For the lower bound this was accomplished by finding the first cell with $\bar{h}_j \le h_1 + \frac{9}{10} \left(h_2 - h_1\right)$ at the final time which we call $x_\text{lower}$, and for the upper bound this was achieved by finding the first cell with $\bar{h}_j \le h_1 + \frac{1}{10} \left(h_2 - h_1\right)$ denoted $x_\text{upper}$. Consequently, the lower bound for the shock speed $S^*_{2,-} = x_\text{lower} / t$ and the upper bound for the shock speed $S^*_{2,+} = x_\text{upper} / t$ were calculated and compared to the analytic value $S_2$ given by \eqref{eq:SWWEMiddleState_S2} in Figure \ref{Fig:DB_ShockSpeed_Comp}. This figure demonstrates that as $\Delta x$ decreases the numerical solutions are better resolving the shock. Additionally it demonstrates that $x_\text{lower}$ and $x_\text{upper}$ provide a bound for the true location of the shock as expected. 
%
\begin{figure}
	\centering
	\includegraphics[width=0.5\textwidth]{./Figures/Simulations/Validation/DBSWWE/ShockSpeedEstimates.pdf}
	\caption{Plot of $S^*_{2,+} / S_2$ (\crosst{blue}),  $S^*_{2,-} / S_2$ (\crosst{red}) and analytic value (\dashedrule) as $\Delta x$ varies.}
	\label{Fig:DB_ShockSpeed_Comp}
\end{figure}

These results demonstrate that the analytic solution of the SWWE has been accurately reproduced by the numerical method. 

\subsection{Forced Solutions}
There are no currently known analytic solutions to the gSGN equations for other $\beta$ values. Hence, to demonstrate the validity and versatility of the method to solve the gSGN for other $\beta$ values, forced solutions are necessary. It is vital for the gSGN equations in particular because for the $\beta$ values tested above $\beta_2$ is zero, and thus the shown analytic solutions do not assess the numerical methods accuracy for the $\beta_2$ term. 

To generate a forced solution the forced gSGN are considered
\begin{subequations}
	\begin{gather}
	\dfrac{\partial h}{\partial t} + \dfrac{\partial (uh)}{\partial x} = \dfrac{\partial h^*}{\partial t} + \dfrac{\partial (u^*h^*)}{\partial x} 
	\label{eq:gSGN_Gh_Forced}
	\end{gather}
	\begin{multline}
	\dfrac{\partial G }{\partial t}  + \dfrac{\partial}{\partial x} \left ( uG + \dfrac{gh^2}{2} - \beta_1 h^3\dfrac{\partial u}{\partial x}\dfrac{\partial u}{\partial x}  - \frac{1}{2} \beta_2 g h^2  \left[h\frac{\partial^2 h}{\partial x^2} + \frac{1}{2}\frac{\partial h}{\partial x}\frac{\partial h}{\partial x}\right]\right ) = \\ \dfrac{\partial G^* }{\partial t}  + \dfrac{\partial}{\partial x} \left ( u^*G^* + \dfrac{g\left(h^*\right)^2}{2} - \beta_1\left(h^*\right)^3\dfrac{\partial u^*}{\partial x}\dfrac{\partial u^*}{\partial x}  - \frac{1}{2} \beta_2 g \left(h^*\right)^2  \left[h^*\frac{\partial^2 h^*}{\partial x^2} + \frac{1}{2}\frac{\partial h^*}{\partial x}\frac{\partial h^*}{\partial x}\right]\right ).
	\label{eq:gSGN_GG_Forced}
	\end{multline}
	\label{eq:gSGN_Forced}
\end{subequations}
The forced gSGN admit the solutions $h^*$, $u^*$ and $G^*$ assuming $G^*$ appropriately satisfies \eqref{eq:G_divergent}. Since these equations are satisfied for any chosen $h^*$, $u^*$ and $G^*$ and any $\beta$ values, forced solutions can be used to verify the method for a larger class of problems than permitted by currently known analytic solutions. In particular, problems were $\beta_2 \neq 0$ which were not covered in the above validation using analytic solutions. Since the left hand-side of these modified equations are approximated by the numerical method, combining the numerical method with the analytic expressions for the right hand-side, produces a method that approximates the forced gSGN equation \eqref{eq:gSGN_Forced} with the same convergence properties as the underlying numerical method for the gSGN equations. 

The following forced solution
\begin{subequations}
	\begin{equation}
	h^*(x,t) = a_0 + a_1 \exp\left( \dfrac{\left(x - a_2 t\right)^2}{2 a_3} \right)
	\end{equation}
	\begin{equation}
	u^*(x,t) = a_4 \exp\left( \dfrac{\left(x - a_2 t\right)^2}{2 a_3} \right)
	\end{equation}
	\begin{align}
	\beta_1(x,t) &= a_6 \\
	\beta_2(x,t) &= a_7
	\end{align}
\end{subequations}
where $G^*$ is given by \eqref{eq:G_divergent}, were used. These forced solutions describe Gaussian bumps in $h$ and $u$ that travel at a constant speed $a_2$. This forced solution was chosen because it is smooth and because the terms in \eqref{eq:gSGN_Forced} are not constant over the whole domain. Smoothness is necessary for the current description of the forced solutions, since it requires the derivatives to be defined in the classical strong sense. This ensures that the forced solutions include all the terms in the gSGN when assessing the numerical method. 

The particular parameter values $a_0=1$, $a_1=0.5$, $a_3=20$ and $a_4=0.3$ were chosen in this investigation, while multiple $\beta$ values were tested we will be focusing on $a_6 = 2/15 + 2/3$ and $a_7=2/15$ below. 

The numerical solutions were produced over the domain $\left[-100m,100m\right]$ with a final time of $t=10s$. The spatial resolution was varied like so $\Delta x = 200 / (100 \times 2^{l})$, while the CFL condition was satisfied by setting $\Delta t = \Delta x  / \left( 2 \left[a_4 + a_2+ \sqrt{g \left(a_0 + a_1\right)}\right] \right)$. The acceleration due to gravity $g=9.81m^2/s$ was used. For the forced solutions the limiting on the reconstruction on $h$, $u$ and $G$ was removed. Because these forced solutions are smooth, such reconstruction is not necessary.

Figure \ref{Fig:FS_Ex} compares an example numerical solution at the final time for $h$, $G$ and $u$ with $\Delta x = 200 / (100 \times 2^{5}) \approx 0.06m$ with the analytic solution. These example solutions demonstrate that the numerical method is able to reproduce the forced solution well, validating the numerical methods approximation to all terms in \eqref{eq:gSGN_G}. 
%
\begin{figure}
	\centering
	\begin{subfigure}{0.32\textwidth}
		\centering
		\includegraphics[width=\textwidth]{./Figures/Simulations/Validation/Forced/iSGN/h.pdf}
		\caption{$h$}
	\end{subfigure}
	\begin{subfigure}{0.32\textwidth}
		\centering
		\includegraphics[width=\textwidth]{./Figures/Simulations/Validation/Forced/iSGN//G.pdf}
		\caption{$G$}
	\end{subfigure}
	\begin{subfigure}{0.32\textwidth}
		\centering
		\includegraphics[width=\textwidth]{./Figures/Simulations/Validation/Forced/iSGN/u.pdf}
		\caption{$u$}
	\end{subfigure}
	\caption{Example plots of initial conditions (\solidrule), analytic solution ({\color{blue}\solidrule}), and numerical solution with $\Delta x \approx 0.06m$ (\tikzcircle{red}).}
	\label{Fig:FS_Ex}
\end{figure}

Figure \ref{Fig:FS_Conv} demonstrates the convergence of the numerical scheme as $\Delta x$ decreases. All quantities of interest are converging at the expected second-order. Since the right hand-sides of \eqref{eq:gSGN_Forced} are evaluated analytically, the observed error is caused by the numerical method. Therefore, these results demonstrate that the scheme is second-order for all terms in the gSGN equations.
%
\begin{figure}
	\centering
	\includegraphics[width=0.49\textwidth]{./Figures/Simulations/Validation/Forced/iSGN/NormResults.pdf}
	\caption{Convergence plot of $h$ (\squaret{blue}) , $G$ (\circlet{red}), $u$ (\trianglet{black}) for the forced solutions for various $\Delta x$ values.}
	\label{Fig:FS_Conv}
\end{figure}


\subsection{Numerical Comparison of Different $\beta$ values}
The validation results in Sections 5.2 and 5.3 demonstrate the ability of the numerical method to accurately solve the gSGN for smooth solutions and in the presence of discontinuities when $\beta_1 = \beta_2 = 0$ where the gSGN reduce to the SWWE. We will now illustrate the different behaviours of this family of equations using numerical solutions to the dam-break problem \eqref{eqn:DB_Init} described in Section 5.2 at $t=35s$. Figure \ref{Fig:DBgSGN_Ex} demonstrates the numerical solutions to $h$ for a variety of $\beta$ values with Figure \ref{Fig:DBgSGN_Ex_Beta} showing the location of the numerical solutions on the plot of $\beta$ values shown above in Figure \ref{Fig:WaveSpeedReg}. 

These numerical solutions are well resolved and compare well to analytic solutions for the SWWE, and the numerical study of the dam-break problem performed by \citet{Pitt-2018-61} for the SGN. By comparing the numerical solution to its location in the $\beta$ plot we can see that the behaviour of solutions is described well by the chain of inequalities \eqref{eq:wavespeedbound1} when $\beta_1 \ge \beta_2$ and \eqref{eq:wavespeedbound2} when $\beta_1 < \beta_2$. In particular, for the SWWE and the regularised SWWE $\beta_1 = \beta_2$ which results in a phase speed which is independent of $k$ and thus a non-dispersive wave model. Consequently, we observe no wave train between the shock and the rarefaction fan in Figures \ref{Fig:DBgSGN_Ex_SWWE} and \ref{Fig:DBgSGN_Ex_rSWWE}. For the dispersive wave models such as the improved dispersion SGN equations with $\mathcal{O}(k^6)$ accuracy in the dispersion relationship and the classical SGN equations with $\mathcal{O}(k^4)$ accuracy in the dispersion relationship, we observe a dispersive wave train between the shock and the rarefaction fan. While for an illustrative example in Region 2 of the $\beta$ plot we observe dispersive wave trains advancing behind the rarefaction fan and ahead of the shock due to the flipping of the chain of inequalities when $\beta_2 > \beta_1$ in \eqref{eq:wavespeedbound2}. 

Interestingly in Figure \ref{Fig:DBgSGN_Ex_iSGN} although the dispersion relationship of the improved dispersion relationship is $\mathcal{O}(k^6)$ accurate, there is a trade-off for large $k$ values, as can be seen by the appearance of a non-physical flat region between the two dispersive wave trains. This behaviour is an artefact of the way we have improved the dispersion relationship of the gSGN and does not agree with the linear theory of water waves acting under gravity. 

\begin{figure}
	\centering
	\begin{subfigure}{0.45\textwidth}
		\centering
		\includegraphics[width=\textwidth]{./Figures/Simulations/Comparison/hEx03.pdf}
		\caption{SWWE ($\beta_1  = \beta_2 = 0$)}
		\label{Fig:DBgSGN_Ex_SWWE}
	\end{subfigure}
	\begin{subfigure}{0.45\textwidth}
		\centering
		\includegraphics[width=\textwidth]{./Figures/Simulations/Comparison/hEx02.pdf}
		\caption{Regularised SWWE ($\beta_1  = \beta_2 = 1$)}
		\label{Fig:DBgSGN_Ex_rSWWE}
	\end{subfigure}
	\begin{subfigure}{0.45\textwidth}
		\centering
		\includegraphics[width=\textwidth]{./Figures/Simulations/Comparison/hEx01.pdf}
		\caption{Improved SGN ($\beta_1  = \beta_2  +  2/3$, $\beta_2 = 2/15$)}
		\label{Fig:DBgSGN_Ex_iSGN}
	\end{subfigure}
	\begin{subfigure}{0.45\textwidth}
		\centering
		\includegraphics[width=\textwidth]{./Figures/Simulations/Comparison/hEx00.pdf}
		\caption{SGN ($\beta_1  = \beta_2  +  2/3$, $\beta_2 = 0$)}
		\label{Fig:DBgSGN_Ex_SGN}
	\end{subfigure}
	\begin{subfigure}{0.45\textwidth}
	\centering
	\includegraphics[width=\textwidth]{./Figures/Simulations/Comparison/hEx04.pdf}
	\caption{Region 2 ($\beta_1  = 1/3$, $\beta_2 = 2/3$)}
	\label{Fig:DBgSGN_Ex_Reg2}
	\end{subfigure}
	\begin{subfigure}{0.45\textwidth}
	\centering
	\includegraphics[width=0.95\textwidth]{./Figures/Explanation/BetaPlotAllExamples.pdf}
	\caption{Location of equations of figures on $\beta$ value plot as in Figure \ref{Fig:WaveSpeedReg} }
	\label{Fig:DBgSGN_Ex_Beta}
	\end{subfigure}
	\caption{Water profile for numerical solution to dam-break problem \eqref{eqn:DB_Init} with $\Delta x = 500 / (100 \times 2^8) \approx 0.19 m$ at $t = 35s$ for different $\beta$ values.}
	\label{Fig:DBgSGN_Ex}
\end{figure}

\section{Conclusion}
A modified version of the numerical scheme for the SGN outlined by \citet{Zoppou-etal-2017} was used to solve the gSGN \cite{Clamond-Dutykh-2018-237,Clamond-et.al-2017-245}. This numerical scheme for the gSGN was validated by describing a fully second-order order implementation as an example numerical method. This example numerical method was validated against analytic solutions of the SGN and SWWE and forced solutions. The analytic solutions demonstrate that the gSGN solver accurately reproduces important members of the gSGN family of equations whilst conserving the quantities of interest, and the forced solutions demonstrate that the method remains second-order for all values of the free parameters, $\beta_1$ and $\beta_2$. The gSGN method described above is the first validated numerical method for the gSGN.
 
\section{References}
\bibliographystyle{unsrtnat}
\bibliography{Bibliography}


\end{document} 