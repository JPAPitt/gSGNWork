\documentclass[10pt]{article}
\usepackage{natbib}
\begin{document}
	\title{Numerical Scheme for the Generalised Serre-Green-Naghdi Model - Reviewer Response}
	
	\maketitle

\section{Introduction}

We kindly thank the reviewers for their constructive advice on the paper and will address their concerns below.
	
\section{Reviewer 1}
\subsection{Comments}
The manuscript is if high quality and after the revision it can be accepted for publication.
\subsection{Response}
Thanks!

This review requires no alterations to the paper.


\section{Reviewer 2}
\subsection{Comments}
Authors made some good improvement. They added some test cases with Experimental case.

\begin{enumerate}
	\item To be blank, i didn't see which is better in Fig.11. 
	Blue line and Black line, 
	i.e. classical and improved SGN , 
	i couldn't distinguish which is better.
	\item  Still, i think the motivation is not well written or written clearly.
	
	If motivation is the new scheme ,
	then as authors states in the reply,
	" other numerical methods relying on smoothess of solutions such as the
	spectral method of Dutykh et al. (2018) for the regularised SWWE sub-family required filtering of the solutions to remove oscillations introduced
	by the method"
	
	I suggest that authors should write these clearly into the paper, and provide the evidence or citation , to show there are stable problems or accuracy problems in the
	spectral method of Dutykh et al. (2018).
	
	Then as the motivation, authors try to give a new scheme. 
	
	Comparison should be provided between new scheme and spectral method of Dutykh et al. (2018) or other published method.
	To show the new scheme is better than others.
	
	If only shows new scheme is correct, without showing the advantages compared with other published method, then people may still use published others' method.
	
	Above things are suggested to be shown in abstract, conclusions and introduction part, to show the contribution of this paper clearly.
\end{enumerate}



\subsection{Response}
\subsubsection{Issue 1}
We have split Figure 11 into Figures 11 and 12, adding panels that zoom in on specific areas. Both these changes allow the black and blue lines to be distinguished better and thus compared. The supporting discussion has also been updated appropriately to match the altered figures.

\subsubsection{Issue 2}
We have have improved the motivation for the paper as communicated by the abstract, introduction and conclusions. 

While the suggested motivations were helpful and illustrated the reviewers issue with the previous motivation, the updated papers motivation is not wholly taken from their suggestions and is instead:

example.

this co
\end{document}